\chapter*{Introduction Générale}
\addcontentsline{toc}{chapter}{Introduction Générale}

Dans un contexte de transformation numérique accélérée, le secteur de la location de voitures en Tunisie fait face à des défis majeurs de modernisation et d'optimisation. La digitalisation des services devient une nécessité pour répondre aux attentes croissantes des clients et améliorer l'efficacité opérationnelle des agences de location.

\section*{Contexte}
Le marché tunisien de la location de voitures se caractérise par :
\begin{itemize}
    \item Une forte saisonnalité liée au tourisme
    \item Une diversité d'acteurs (grandes enseignes et petites agences locales)
    \item Des processus souvent manuels et peu optimisés
    \item Une demande croissante de services numériques
\end{itemize}

\section*{Problématique}
Les principaux défis identifiés sont :
\begin{itemize}
    \item La difficulté pour les clients de comparer les offres
    \item Le manque de visibilité en temps réel sur la disponibilité des véhicules
    \item La complexité des processus de réservation et de gestion
    \item L'absence de solutions adaptées au contexte local
\end{itemize}

\section*{Objectifs}
Ce projet vise à développer une plateforme innovante pour :
\begin{itemize}
    \item Digitaliser le processus de location de voitures
    \item Faciliter la mise en relation entre agences et clients
    \item Optimiser la gestion des flottes et des réservations
    \item Améliorer l'expérience utilisateur grâce à la géolocalisation
\end{itemize}

\section*{Méthodologie}
La réalisation du projet s'appuie sur :
\begin{itemize}
    \item Une analyse approfondie des besoins du marché
    \item Le choix de technologies modernes et éprouvées
    \item Une approche itérative et incrémentale
    \item Des tests rigoureux à chaque étape
\end{itemize}

\section*{Technologies Utilisées}
La solution repose sur un stack technologique robuste :
\begin{itemize}
    \item \textbf{Backend} : Framework Django avec Python
    \item \textbf{Base de données} : PostgreSQL avec extension PostGIS
    \item \textbf{Frontend} : HTML5, CSS3, JavaScript moderne
    \item \textbf{Outils} : Git, Docker, Django TestCase
\end{itemize}

\section*{Structure du Rapport}
Ce rapport s'articule autour de cinq chapitres :

\begin{enumerate}
    \item \textbf{Présentation du Projet} : Contexte, enjeux et objectifs détaillés
    
    \item \textbf{Analyse des Besoins} : Spécifications fonctionnelles et techniques
    
    \item \textbf{Architecture du Système} : Conception et implémentation
    
    \item \textbf{Tests et Validation} : Stratégie de test et résultats
    
    \item \textbf{Conclusion et Perspectives} : Bilan et évolutions futures
\end{enumerate}

\section*{Impact Attendu}
La plateforme développée vise à apporter :
\begin{itemize}
    \item Une modernisation du secteur de la location de voitures
    \item Une meilleure expérience client
    \item Une optimisation des processus métier
    \item Une base pour l'innovation future
\end{itemize}

Cette introduction présente le cadre général du projet et les éléments clés qui seront développés dans les chapitres suivants. L'objectif est de fournir une solution complète et adaptée aux besoins spécifiques du marché tunisien de la location de voitures, tout en s'appuyant sur des technologies modernes et évolutives.




